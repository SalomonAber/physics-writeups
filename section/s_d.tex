\section{Resonance Tubes}
CPACs 1, 3 and 4
\hfill
\nth{27} November 2019

\subsection{Method}
\begin{enumerate}
  \item Record the background count over 1000 seconds and calculate counts per second, I.
  \item Set up the apparatus as shown in the diagram.
  \item The Geiger tube should be at a fixed distance of 20cm from the source and this distance should not be altered during the experiment.
  \item Record the count rate, $I_0$, with no absorber present.
  \item Use a pair of Vernier callipers or a micrometer screw gauge to record the thickness, x, of one of the absorbers.
  \item Plot a graph of $\ln{I}$ against x.
  \item Determine the absorption coefficient, $\mu$, for Aluminium from your graph and the equation $\ln{I}=\ln{I_0} - \mu \cdot x$
\end{enumerate}

\subsection{Safety}
To reduce the risk of the beta source damaging tissue, the beta source was directed at a wall. A sign was also erected to warn other students to keep clear of the surrounding area.

\subsection{Results}

The average background radiation was calculated to be:

\begin{equation}
  \frac{368}{1000} = 0.368 cps
\end{equation}
This value was subtracted from each count rate measured.

\begin{center}
  \pgfplotstabletypeset[
    columns={t,i1, i2, a},
    columns/t/.style={column type = |p{2.0cm},column name= Absorber thickness (mm)},
    columns/i1/.style={column type = |p{2.0cm}|,column name= $I_1$ (cps)},
    columns/i2/.style={column type = p{2.0cm}|,column name= $I_2$ (cps)},
    columns/a/.style={column type = p{2.1cm}|,column name= $I_{average}$ (cps)},
    string type
  ]{data/r_d_1.txt}
\end{center}

\subsection{Graph}

\begin{figure}[H]
  \centering
  \begin{tikzpicture}
    \begin{axis}[ylabel={Natural log of count rate, $\ln{I}$},xlabel={Absorber thickness, x}]
      \addplot [mark = *] table [x=tp,y=la]{data/r_d_1.txt};
      \addplot [thick, red] table [
        x=tp,
        y={create col/linear regression={y=la, x=tp}}
      ]{data/r_d_1.txt};
    \end{axis}
    \xdef\regaD{\pgfplotstableregressiona}
    \xdef\regbD{\pgfplotstableregressionb}
  \end{tikzpicture}
\end{figure}

\subsection{Analysis}
The gradient of the graph is \regaD. By consulting the equation mentioned in the method we can conclude that the value of $\mu$ is 65 to three significant figures. As we could not find a trusted value for $\mu$, no comparisons can be made. It can however be concluded that the results of the experiment are untrustworthy as they do not appear to have a linear relationship. A possible justification for the fact that the count rate at a $3.12mm$ thickness is lower than the background count rate is that the aluminium is also absorbing some background radiation.


\subsection{Conclusion}
From the results measured it can be concluded that the experiment roughly followed the trend expected.
To increase the accuracy of the experiment the count rate should be recorded over a 200 second period.
Furthermore, the distance between the source and detector should be increased.
