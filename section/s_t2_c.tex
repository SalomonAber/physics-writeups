\section{Search Coil Investigation}
CPACs 1, 2, 4 and 5
\hfill
\nth{27} January 2020

\subsection{Abstract}
The aim of this experiment is to investigate the magnetic field produced by a Helmholtz coil as well as the respective emf induced in a Search coil.

\subsection{Equiptment}
\begin{itemize}
\item Search coil magnetometer
\item Helmholtz coil
\item Signal generator
\item Oscilloscope
\item Teltron Stand
\end{itemize}

\subsection{Methods}

\subsubsection{Distance Method}
\begin{enumerate}
    \item Connect the Helmholtz coil to the signal generator.
    \item Connect the search coil to the oscilloscope.
    \item Using clamp stands position the search coil in the center of the Helmholtz coil such that maximum emf is induced.
    \item Move the search coil along the axis through the Helmholtz coil by increments of 1cm for the first 10cm and increments of 2cm after that.
    \item Record the maximum voltage displayed on the oscilloscope for each distance.
    \item Plot a graph of distance, x, against voltage, v.
\end{enumerate}

\subsubsection{Angle Method}
\begin{enumerate}
    \item Connect the Helmholtz coil to the signal generator.
    \item Connect the search coil to the oscilloscope.
    \item Using clamp stands position the search coil in the center of the Helmholtz coil such that maximum emf is induced.
    \item Rotate the Helmholtz coil in the z-axis in $10^{\circ}$  increments for values between $-90^{\circ}$ and $90^{\circ}$.
    \item Record the maximum voltage displayed on the oscilloscope for each angle.
    \item Plot a graph of angle, $\theta$ against voltage, v.
\end{enumerate}

\subsection{Derivation}
The formula for the on-axis field due to a Helmholtz coil is given by the Biot-Savart law\cite{wiki:biotsavart} to be:
\begin{equation*}
    B\left(x\right) = \frac{\mu_0  \cdot n  \cdot I  \cdot r^2}{2\left(r^2 + x^2\right)^{3/2}}
\end{equation*}
Where:
\begin{itemize}
    \item $\mu_0$ is the permeability constant
    \item $n$ is the number of turns
    \item $I$ is the coil current
    \item $r$ is the coil radius
    \item $x$ is the coil distance, on-axis, to the point
\end{itemize}

The voltage produced by the signal generator is modelled by:
\begin{equation*}
    V = 20 \cdot \sin \left( \num{5e3} \cdot t \right)
\end{equation*}

The resistance of the Helmholtz coil is estimated to be:
\begin{equation*}
    R = \frac{\rho L}{A} = \frac{\num{1.72e-8} \cdot 320 \cdot 2 \pi \cdot 0.075}{\pi \cdot 0.0003^2} \approx \SI{9.173}{\ohm}
\end{equation*}

As a result, the Helmholtz coil current at a time, $t$, is:
\begin{equation*}
    I \left( t \right) = \frac{V \left( t \right)}{R}=\frac{20 \cdot \sin \left(\num{5e3} \cdot t\right) }{9.173}
\end{equation*}

This gives a magnetic field of:
\begin{equation*}
    B \left( x,t \right) = \frac{\num{0.246658e-5} \cdot \sin \left( \num{5e3} \cdot t \right)}{\left( 0.075^2 + x^2 \right)^{3/2}}
\end{equation*}
 
The induced voltage in the search coil magnetometer is described by Faraday's law of Induction\cite{wiki:faraday}:
\begin{equation*}
    e = - N\frac{\mathrm d \Phi}{\mathrm dt}
\end{equation*}

With N being the number of terms.
By Gauss's law\cite{wiki:gauss} for magnetism, the magnetic flux is:

\begin{equation*}
    \Phi_B = \mathbf{B} \cdot \mathbf{S} = BS \cos \theta
\end{equation*}

Where $\theta$ is the angle between the axes of the Helmholtz coil and Search coil and $\cdot$ is the dot product.
The surface area of the Search coil is:

\begin{equation*}
    S = 2\pi \cdot 0.00508 \cdot 2\pi \cdot 0.005 \cdot 5000 \approx \SI{986.96}{\metre\squared}
\end{equation*}

Taking the derivitive of the magnetic flux with respect to time gives:
\begin{equation*}
    \frac{\mathrm d\Phi_B}{\mathrm dt} = \frac{\num{0.246658e-5} \cdot \sin \left( \num{5e3} \cdot t \right) }{\left( 0.075^2 + x^2 \right)^{3/2}} \cdot \num{5e3} \cdot 986.96 \cdot \cos \theta
\end{equation*}

Hence the maximum voltage measured should be:
\begin{equation*}
    e_{max} \left( x \right) = 320 \cdot \frac{\num{0.246658e-5} }{\left( 0.075^2 + x^2 \right)^{3/2}} \cdot \num{5e3} \cdot 986.96 \cdot \cos \theta
\end{equation*}

\subsection{Results}

In order to match the expected and observed voltages, a correction factor of $\num{5.08e-6}$ was used.

\subsubsection{Tables}

\begin{figure}[H]
    \centering
  \pgfplotstabletypeset[
    columns={x, e, v},
    columns/x/.style={column type = |p{3.5cm},column name= {distance, x (m)}},
    columns/e/.style={column type = |p{3.5cm}|,column name= {expected Voltage, e (V)}},
    columns/v/.style={column type = p{3.5cm}|,column name= {observed Voltage, v (V)}},
    string type
  ]{data/r_t2_c_1.txt}
  \caption{Data from Method I}
\end{figure}

\begin{figure}[H]
    \centering
    \pgfplotstabletypeset[
      columns={t, e, v},
      columns/t/.style={column type = |p{3.5cm},column name= {angle, $\theta$ ($^{\circ}$)}},
      columns/e/.style={column type = |p{3.5cm}|,column name= {expected Voltage, e (V)}},
      columns/v/.style={column type = p{3.5cm}|,column name= {observed Voltage, v (V)}},
      string type
    ]{data/r_t2_c_2.txt}
    \caption{Data from Method II}
\end{figure}

\subsubsection{Graphs}

\begin{figure}[H]
    \centering
    \begin{tikzpicture}
      \begin{axis}[ylabel={voltage, v (V)},xlabel={distance, x (m)}]
        \addplot [mark = *] table [x=x,y=v]{data/r_t2_c_1.txt};
        \addlegendentry{observed}
        \addplot [thick, red] table [x=x,y=e]{data/r_t2_c_1.txt};
        \addlegendentry{expected}
      \end{axis}
    \end{tikzpicture}
    \caption{Graph for Method I}
\end{figure}

\begin{figure}[H]
    \centering
    \begin{tikzpicture}
      \begin{axis}[ylabel={voltage, v (V)},xlabel={angle, $\theta$ ($^{\circ}$)}]
        \addplot [mark = *] table [x=t,y=v]{data/r_t2_c_2.txt};
        \addlegendentry{observed}
        \addplot [thick, red] table [x=t,y=e]{data/r_t2_c_2.txt};
        \addlegendentry{expected}
      \end{axis}
    \end{tikzpicture}
    \caption{Graph for Method II}
\end{figure}

\subsection{Evaluation}
As seen in the graphs above, with the use of a correction factor, our predictions are almost a perfect fit for the data.
The correction factor is small enough to remove the need for any calculation of uncertainty.
The Resistance of the Helmholtz coil, and surface area of the Search coil where cruedly approximated.
This is the most likely reason for the large correction factor.
Although the constants derived where incorrect, the data followed the expected shape therefore partially confirming the Biot-Savart law and Faraday's law of Induction.
This experiment could be improved through the use of more accurate equiptment, i.e. using a digital protractor.
In addition to this, important constants such as the resistance in the coils should be experimentally found instead of being approximated.